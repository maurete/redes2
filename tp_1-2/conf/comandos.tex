% comandos.tex
% en este archivo defino todos los comandos/environment que quiera usar en mi documento.
%
% ===
% === Comandos
% ===
% 
% T: para escribir texto común cuando en modo math
%    uso: \T{texto que aparecerá en letra normal}
\newcommand{\T}{\textrm}
%
% aclaracion: dibuja un recuadrito aclaratorio, como <quote> en HTML.
%             uso: \aclaracion{Texto...}
\newcommand{\aclaracion}[1]{%
\smallpencil\ \begin{minipage}{0.9\textwidth}
\vspace*{6pt}{#1}\smallskip\end{minipage}}
%
% consigna: parecido a aclaración, pero con texto _slanted_
%           uso: \consigna{Consigna...}
\newcommand{\consigna}[1]{%
\leftpointright\ \parbox[t]{0.9\textwidth}{\textsl{#1}\vspace{8pt}}}
%
% pinterno: para representar el producto interno entre los dos argumentos
%           uso: \pinterno{X}{Y}
\newcommand{\pinterno}[2]{%
\left\langle #1 , #2 \right\rangle}
%
% === Estilos de texto
%
% resalt: resaltado con fondo verde
%         uso: \resalt{texto resaltado}
\newcommand{\resalt}{\colorbox{green}}
%
% sfbf: texto en negrita + slanted
%       uso: 
\newcommand{\sfbf}{\bfseries\textsf}
%
% eng: itálica (para palabras en inglés)
%      uso: \eng{some English text}
\newcommand{\eng}{\textit}
%
% mean: significado de una sigla - slanted
%       uso: (...) SNCF: \mean{Société Nationale des Chemins de Fer Francais} ...
\newcommand{\mean}{\textsl}
\newcommand{\desc}{\textsl}
%
% defin: pone en negrita el texto, útil para definiciones
%        uso: \defin{asshole}: vulgar slang for anus
\newcommand{\defin}{\textbf}
%
% R, N: cambia la tipografía en modo math, probar para ver cómo quedan
%       uso: \R{R} , \N{N}
\newcommand{\R}{\mathds}
\newcommand{\N}{\mathbf}
%
% lil: para escribir texto pequeño. más cómodo que { \footnotesize texto pequeño... }
%      uso: \lil{texto pequeño... }
\newcommand{\lil}[1]{\footnotesize #1}  %Para texto pequeñooo
%
% mono: escribe el texto que le paso como parámetro con letra de ancho fijo
%       uso: \mono{texto monoespaciado}
\newcommand{\mono}[1]{{\tt #1}}
%
% === Símbolos
%
\newcommand{\y}{\wedge}			%Y (Lógica)
\newcommand{\ve}{\vee}			%O (Lógica)
\newcommand{\ent}{\supset}		%Entonces (Lógica)
\newcommand{\dimp}{\leftrightarrow}	%Doble implicativo, equivalencia (Lógica)
\newcommand{\sii}{\leftrightarrow}	%Si y sólo si (Lógica)
\newcommand{\equi}{\equiv}		%Equivalencia (Lógica)
\newcommand{\portanto}{\vdash}		%Por lo tanto (Lógica)
\newcommand{\por}{\cdot}		%Producto punto
\newcommand{\RR}[1][1]{\mathds{R}}	%R de reales
%
%
% ===
% === Environments
% ===
% 
% enunciado: un environment que básicamente tiene el mismo efecto que el
%            comando consigna.
%            uso: \begin{enunciado} ... contenido ... \end{enunciado}
\newenvironment{enunciado}
{\leftpointright\ \begin{varwidth}[t]{0.9\textwidth}\textsl}
{\end{varwidth}\vspace{8pt}}
%
% pvi: para tipear la definición de un problema de valor inicial/funciones
%      definidas de a trozos/etc directamente en el texto (sin necesidad de
%      cambiar a un modo matemático.
%      uso: \begin{pvi} linea1 \\ linea 2 \\ ... \end{pvi}
\newenvironment{pvi}{\begin{equation*}\begin{cases}}
{\end{cases}\end{equation*}}
%
% verbatimsmall: un verbatim con letra más chica. usualmente queda bastante
%                mejor que el verbatim pelado.
%                uso: \begin{verbatimsmall} ........ \end{verbatimsmall}
\newenvironment{verbatimsmall}{\small\begin{verbatim*}}
{\end{verbatim*}}
%
%
% ===
% === Comandos ``históricos''
% ===
%
%% %\begin{pspicture}
%% \def\tierra(#1){%Para dibujar el símbolo de tierra en el entorno PSTricks
%% 	\rput(#1){
%% 		\psdot(0,0)
%% 		\psline(0,0)(0,-0.45)
%% 		\psline(-0.5,-0.45)(0.5,-0.45)
%% 		\psline(-0.35,-0.6)(0.35,-0.6)
%% 		\psline(-0.2,-0.75)(0.2,-0.75)
%% 	}%
%% }
%% %\end{pspicture}

%% \newcommand{\codigo}[2]{%Para generar un recuadro con código
%% 	%\setlength{\hrulewidth}{0.1pt}
%% 	\begin{flushleft}
%% 	\underline{#1}
%% 	\begin{tabular}{@{\quad}|l}
%% 		\begin{minipage}{.85\textwidth}\smallskip{#2}
%% 	\end{minipage}\end{tabular}\end{flushleft}%
%% }

%% \newcommand{\filecodigo}[1]{%Insertar código verbatim desde un archivo
%% \codigo{#1}{\verbatiminput{#1}}}%Requiere el paquete verbatim
%% \newcommand{\filecodigobis}[1]{{\verbatiminput{#1}}}%Requiere el paquete verbatim

%% \newcommand{\grafico}[3][1]{%Para generar un plot de un archivo con coords.
%% %\def\deequis=#1
%% \begin{minipage}{0.5\textwidth}\begin{center}
%% \begin{pspicture}(6,5)
%% 	\psgrid[subgriddiv=1,gridlabels=0pt,gridwidth=.1pt](1,3)(1,1)(6,5)
%% 	\psset{xunit=5cm,yunit=2cm}
%% 	\fileplot[linewidth=1pt,linecolor=blue,origin={0.2,1.5}]{#2}
%% 	\psset{xunit=1cm,yunit=1cm}
%% 	\psaxes[Dx=#1,dx=5,Oy=-1,Dy=1,dy=2]{-}(0.9,1)(6,5)
%% 	\rput(4,0.4){\textsl{#3}}
%% \end{pspicture}\end{center}\end{minipage}}

%% \newcommand{\eqncode}[2]{%
%% \begin{center}
%% \begin{tabular}{l@{\hspace{0.5cm}}r}
%% \begin{minipage}{.4\textwidth}
%% \begin{equation*}
%% #1
%% \end{equation*}
%% \end{minipage}
%% &
%% \fbox{\begin{minipage}{.4\textwidth}
%% %\setlength{\parskip}{4mm}
%% \filecodigobis{#2}
%% \end{minipage}}
%% \end{tabular}
%% \end{center}
%% }

%% \newcommand{\eqncodeb}[2]{%
%% \begin{center}\begin{tabular}{l@{\hspace{0.5cm}}r}
%% \begin{minipage}{.4\textwidth}#1\end{minipage} &
%% \fbox{\begin{minipage}{.4\textwidth}\filecodigobis{#2}\end{minipage}}
%% \end{tabular}\end{center}}

%% \newenvironment{matemcode}[1]{\newline
%% \begin{tabular}{l@{\hspace{0.5cm}}r}
%% \begin{minipage}{.4\textwidth}
%% \parbox[t]{.4\textwidth}{\begin{equation*}#1\end{equation*}}\end{minipage}
%% &\begin{Sbox}\begin{minipage}{.4\textwidth}}
%% {\end{minipage}\end{Sbox}\fbox{\TheSbox}\end{tabular}\newline}

%% \newenvironment{encuadrar}[1]{\begin{Sbox}\begin{varwidth}{#1\textwidth}}
%% {\end{varwidth}\end{Sbox}\fbox{\TheSbox}}

%% \newenvironment{parboxenv}{\begin{Sbox}}
%% {\end{Sbox}\parbox[t]{.9\textwidth}{\TheSbox}}
